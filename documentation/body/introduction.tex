\newpage
\chapter*{Introduction}
\addcontentsline{toc}{chapter}{Introduction}
%This is typically an outline description detailing the background to the problem.

%
%\newglossaryentry{DAM}
%{
%  name={DAM},
%  description={decameter radio emissions in the 10-100 m wavelengths},
%  sort=DAM
%}
%

\gls{UI}, and \gls{CSS} frameworks are abundant and readily available and generally covered by permissible licence agreements \citep{CODY16}. This not only allows commercial use of the frameworks but ultimately encourages tie in to a particular framework. End users expect all websites to work across their devices, laptops, tablets and phones. This requirement of responsive web applications ensures that hand coding a solution is a daunting task for entry level developers.




%
\newglossaryentry{CSS}
{
  name={CSS},
  description={Cascading Stylesheets},
  sort=CSS
}
%


%
\newglossaryentry{UI}
{
  name={UI},
  description={User Interface},
  sort=UI
}
%

Using a framework allows the user to speed up the initial mock-up process, they offer clarification on common \gls{CSS} issues, and have wide browser support. Frameworks are good for responsive design, offer clean, and tidy code. There are disadvantages to using frameworks such as an abundance of unused code left over, this is seen more so in large front end frameworks such as Twitter Bootstrap. There is a slower learning curve with using frameworks, as the majority of work is done for them. All the user does is change small features such as colour, not encouraging them to learn through development.

Frameworks such as Twitter Bootstrap \citep{SASS16}, and Zurb Foundation \citep{LESS16} offer complete solutions, with ready built code for forms, buttons, fluid layouts, and popovers. Skeleton \citep{SKEL16} is an example of the other end of the spectrum. Skeleton is a boilerplate for responsive, and mobile first development. It is designed to be light, and is built with less than 400 lines of code. Unlike Bootstrap or Foundation, skeleton is designed to be the users starting point, not their full solution. 

This project has been built to offer the end user a complete solution designed with an entry level user in mind. The framework is non-opinionated unlike frameworks such as Bootstrap which is very opinionated about their design \citet{KEMH16}. This framework is designed to be the starting point for a user, for the user to manipulate, and build upon.

Instead of the user building large complex websites with repeating code, one of the project goals is to get the user familiar with concepts such boilerplate templates, partials and a \gls{CLI}. Partials break up the html code into smaller more manageable fragments that can be used across multiple html files. The framework has been built with this in mind, creating classes, and id's that can be reused instead of having a bloated package with a lot of unused code \citep{KAR15}.

\newglossaryentry{CLI}
{
  name={CLI},
  description={Command Line Interface},
  sort=CLI
}

A major benefit of this project, is the ability to customise or use prebuilt themes also known as boilerplate templates. By generating the templates for the user, it can allow them to concentrate on the aesthetics more so than the structure of the project. Using templates does have its negatives, the templates themselves can be expensive costing hundreds if not thousands as seen on Themeforest \citep{THEME17}. Many shortcuts could be used in the development of boilerplate, not adhering to W3S standards. These boilerplate's can also offer little customisation \citep{NATH16}.

Further attempts to incorporate Bootstrap into projects demonstrated the syntax as very unfriendly, and noticeably more difficult than hand coded \gls{CSS}. Bootstrap syntax such as: \begin{lstlisting}
	 <div class="col-sm-4">\end{lstlisting} 
	 does not describe in any form that it would be displayed as a three column layout in the browser. Based on the snippet above, an entry level user is unlikely to know how to change this from a three column layout to a single column layout. 
	 
Additionally while working as a web developer mentor at Coderdojo, students were observed to have similar experiences. Many were reluctant to learn frameworks such as Bootstrap as they were too confusing.

%
\section*{License}
\addcontentsline{toc}{section}{License}
The application will be released under the MIT license following in the footsteps of the other frameworks researched for this project. The MIT license is permissive, allowing permissions such as commercial use, private use, distribution, and modification. With the MIT license, another user cannot claim the work as their own, derivatives are allowed as long as the original author is credited, and the original authors cannot be held liable.

