\newpage
\chapter*{Introduction}
\addcontentsline{toc}{chapter}{Introduction}
\section*{Background}
\addcontentsline{toc}{section}{0.1 Background}

\gls{UI}, and \gls{CSS} frameworks are abundant and readily available and generally covered by permissible licence agreements \citep{CODY16}. This not only allows commercial use of the frameworks but ultimately encourages tie in to a particular framework. End users expect all websites to work across their devices, laptops, tablets and phones. This requirement of responsive web applications ensures that hand coding a solution is a daunting task for entry level developers.

%
\newglossaryentry{CSS}
{
  name={CSS},
  description={Cascading Stylesheets},
  sort=CSS
}
%


%
\newglossaryentry{UI}
{
  name={UI},
  description={User Interface},
  sort=UI
}
%

\newglossaryentry{HTML}
{
  name={HTML},
  description={HyperText Markup Language},
  sort=HTML
}

Using a framework allows the user to speed up the initial mock-up process, they offer clarification on common \gls{CSS} issues, and have wide browser support. Frameworks are good for responsive design, offer clean, and tidy code. There are disadvantages to using frameworks such as an abundance of unused code left over, this is seen more so in large front end frameworks such as Twitter Bootstrap. There is a slower learning curve with using frameworks, as the majority of work is done for them, users tend to make only minor changes such as modifications to colour.

Frameworks such as Twitter Bootstrap \citep*{SASS16}, and Zurb Foundation \citep*{LESS16} offer complete solutions, with ready built code for forms, buttons, fluid layouts, and popovers. Skeleton is an example of the other end of the spectrum. Skeleton is a boilerplate for responsive, and mobile first development. It is designed to be light, and is built with less than 400 lines of code \citep{SKEL16}. Unlike Bootstrap or Foundation, Skeleton is designed to be the users starting point, not their full solution. 

Taylor'd UI has been designed and built with the intension of offering a complete solution aimed at an entry level user. The framework is non-opinionated unlike frameworks such as Bootstrap which is very opinionated about their design \citep{KEMH16}. This framework is designed to be the starting point which a user may then manipulate, and build upon.

Rather than building large complex websites with repeating code, one of the project goals is to allow a safe environment where a user can experiment with templates, partials. Partials break up the \gls{HTML} code into smaller more manageable fragments that can be used across multiple \gls{HTML} files. The framework has been built with this in mind, creating classes, and id's that can be reused instead of having a bloated package with a lot of unused code \citep{KAR15}.

\newglossaryentry{CLI}
{
  name={CLI},
  description={Command Line Interface},
  sort=CLI
}

Prebuilt code such as standard default themes are often referred to as a template. A major benefit of this project is to enable the easy modification of such templates. By generating the templates for the user, it can allow them to concentrate on the aesthetics more so than the structure of the project. Using prebuilt themes will not suit every situation. The templates themselves can be expensive as can be seen on Themeforest \citep{THEME17}, and might not adhere to the W3C standards, or worse might offer little customisation \citep{NATH16}. 

Further attempts to incorporate Bootstrap into projects demonstrated the syntax as very unfriendly, and noticeably more difficult than hand coded \gls{CSS}. Bootstrap syntax such as: \begin{lstlisting}
	 <div class="col-sm-4">\end{lstlisting} 
	 does not describe in any form that it would be displayed as a three column layout in the browser. Based on the snippet above, an entry level user is unlikely to know how to change this from a three column layout to a single column layout. 
	 
Additionally while working as a web developer mentor at Coderdojo, students were observed to have similar experiences. Many were reluctant to learn frameworks such as Bootstrap as they were too confusing. Taylor'd UI could potentially be used in an educational environment to provide students a structured introduction to web development.

The framework will contain similar features from both Bootstrap, and Foundation as seen in table \ref{features} such as tables, lists, breadcrumbs, and pagination. Features such as tooltips, and right to left language support are outside the scope of the project. The hope for the project is to be complete framework but having a smaller file size than both foundation, and Bootstrap borrowing concepts from Skeleton in keeping the framework light, and nimble. Taylor'd UI has five major aspects to it: 
\begin{itemize}
	\item For entry level users
	\item Theme-able
	\item Introduce the concepts of partials
	\item Have templates for the user to manipulate
	\item Install framework from the command line
\end{itemize}
%
\newpage
\section*{License}
\addcontentsline{toc}{section}{0.2 License}
The application will be released under the MIT license following in the footsteps of the other frameworks researched for this project. The MIT license is permissive, allowing permissions such as commercial use, private use, distribution, and modification. With the MIT license, another user cannot claim the work as their own, derivatives are allowed as long as the original author is credited, and the original authors cannot be held liable.

The link to this project can be found at:

https://github.com/MikeZTaylor/Taylord-UI
