\newpage
\chapter*{User Manual}
\addcontentsline{toc}{chapter}{User Manual}
\section*{Introduction}
\addcontentsline{toc}{section}{Introduction}
% Chapter for each research question
% how we validated it and proved it worked

The project is to build a full UI framework that is aimed towards an entry level user. The framework is be non-opinionated, light weight, easy to understand, theme-able, and most importantly easy to learn.

The framework has minimal styling, the author has only added styling where it is needed, and that styling is kept to basic colours. This was to ensure the framework does not carry the author's design opinions. With been non-opinionated the framework is bare so that the end user is encouraged to not rely upon the built in styling, but instead to use it as a starting point to build upon it. The framework is based on a responsive grid system of 12 grids, allowing the end user to build a seamless experience from desktops to mobile devices.

The hope is that the framework will be as small of possible as of the source code deadline, the framework weighs ----- this allows projects using it load faster on mobile devices, and in countries where fast internet is either not available or expensive. 

To stop code from been repeated unnecessary, code sections have been broken into partials allow for the ability for the content to be broken up into manageable pieces, removing receptive code such as headers, and footers. PHP is then used to combine the files into one when loaded in the browser. 

The framework will contain similar features from both Bootstrap, and Foundation as seen in table \ref{features} such as tables, lists, breadcrumbs, and pagination. Features such a tooltips, and right to left are outside the scope of the project. The hope for the project is to be complete framework but having a smaller file size than both foundation, and Bootstrap borrowing concepts from Skeleton in keeping the framework light, and nimble. 

The proposed framework will be built using the three key points mentioned below: 
\begin{itemize}
	\item For Entry Level Users
	\item Themeable
	\item Introduce the concepts of Templates, and Partials
\end{itemize}
