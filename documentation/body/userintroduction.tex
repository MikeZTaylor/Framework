\newpage
\chapter*{User Manual}
\addcontentsline{toc}{chapter}{User Manual}
\section*{Introduction}
\addcontentsline{toc}{section}{Introduction}
% Chapter for each research question
% how we validated it and proved it worked

The project is to build a full \gls{UI} framework that is aimed towards an entry level user. The framework is to be non-opinionated, light weight, easy to understand, theme-able, and most importantly easy to learn.

The framework has minimal styling, the author has only added styling where it is needed, and that styling is kept to basic colours. This was to ensure the framework does not carry the author's design opinions. With been non-opinionated the framework is bare so that the end user is encouraged to not rely upon the built-in styling, but instead to use it as a starting point to build upon it. The framework is based on a responsive grid system of 12 grids, allowing the end user to build a seamless experience from desktops to mobile devices.

One of the aims is to make the library as small as possible to ensure it  loads quickly on mobile devices and in situations where fast network connectivity is not available. This also ensures that the data used on mobile devices is not unnecessarily burdened.

To stop code from being repeated unnecessarily, code sections have been broken into partials to allow for the ability for the content to be broken up into manageable pieces, removing receptive code such as headers, and footers.

\newpage
The framework will contain similar features from both Bootstrap, and Foundation as seen in table \ref{features} such as tables, lists, breadcrumbs, and pagination. Features such as tooltips, and right to left language support are outside the scope of the project. The hope for the project is to be complete framework but having a smaller file size than both foundation, and Bootstrap borrowing concepts from Skeleton in keeping the framework light, and nimble. 

The proposed framework will be built using the five key points mentioned below: 
\begin{itemize}
	\item For entry level users
	\item Theme-able
	\item Introduce the concepts of partials
	\item Have boilerplate templates for the user to use
	\item Have a command line interface to add a theme to a boilerplate template
\end{itemize}
