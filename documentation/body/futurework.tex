\newpage
\chapter*{Future Work}
\addcontentsline{toc}{chapter}{Future Work}

%\begin{itemize}
%  \item sum up the answers to all the questions
%  \item state contributions again
%  \item where my work fits in the grander scheme
%  \item further work
%\end{itemize}


The original plan of the framework was to create a documentation website showcasing the framework, and detailing everything about the framework with code snippets, and examples of usage. When the framework was been developed, a kitchen sink website was created first to ensure that the code was working as it should. Due to time constraints, the finished website hasn't been developed. For the continuation of the project, this website will be built using the framework, and have documentation on each of the core functionality of the framework. 

In continuing on developing the framework, the following features would be added to the framework to make it more robust, and eliminate the need to write extra CSS:
\begin{itemize}
	\item Badges to show unread content or to be used to display notifications
	\item Breadcrumbs to show the user where they are are on a website
	\item Tabs for tabbed navigation
	\item Off canvas sidebar that slides in and out of a page, ideal for mobile design
	\item Modal to create modal dialog boxes
	\item Inverse to reverse the style of any component for light or dark backgrounds
\end{itemize}

In addition to CSS, jQuery features to be added that would help the framework work more seamlessly such as making links active in the navigation bar, ensuring that the footer section always stays at the bottom of the page, regardless of page height, adding close options to the alerts, et al. 

%
%%%%%%%%%%%%%%%%%%%%%%%%%%%%%%%%%%%%%%%%%%%%%%%%%%%%%%%%%%%%%%%%%%%%%%%%%%%%%%%%%%%%%%%%%%%
%

%\subsection*{Creating Spectrograms with HackRF}
%\addcontentsline{toc}{subsection}{Creating Spectrograms with HackRF}
