\newpage
\chapter*{Conclusions}
\addcontentsline{toc}{chapter}{Conclusions}

Overall, I did enjoy developing this project. I feel that I could have worked more on the project but I found it difficult to create a balance between my different modules and the project. 

I feel that I could have worked better on my time management, and project management. To create a balance where all my work got the attention it deserved. 

I did want to work more on the project, but with having four other modules that also required my attention, I was and still am unsure of what takes priority. In the end I found that I was prioritising the modules over the project as their deadlines were scheduled at a sooner date than the project.

In developing my project, I did find out that what I knew of CSS was only the surface. Throughout the development of this project, I was constantly learning about a new feature or component that I could add to my project or that made the development of my project more seamless. 


If I could start my development over, I would start by creating a development plan. Using the plan, I would have set out blocks of time to either do project work or college work, making this semester more efficient. 

I would also build more extra components into the project that would aid the end users development.

 In conclusion, I feel extremely confident in my skills as a Web Developer with a good understanding of Web Technologies that could only be of benefit to me when I graduate college. I would have liked to have had my project more complete but I am proud of what I have accomplished this semester. 
 
 
%\begin{itemize}
%  \item sum up the answers to all the questions
%  \item state contributions again
%  \item where my work fits in the grander scheme
%  \item further work
%\end{itemize}



%
%%%%%%%%%%%%%%%%%%%%%%%%%%%%%%%%%%%%%%%%%%%%%%%%%%%%%%%%%%%%%%%%%%%%%%%%%%%%%%%%%%%%%%%%%%%
%
 

%\subsection*{Creating Spectrograms with HackRF}
%\addcontentsline{toc}{subsection}{Creating Spectrograms with HackRF}
