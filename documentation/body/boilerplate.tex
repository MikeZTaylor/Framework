\newpage
\chapter*{Templates}
\addcontentsline{toc}{chapter}{Template}
% Chapter for each research question
% how we validated it and proved it worked


Template's are a pre-designed webpages, or a set of \gls{HTML} webpages that an end user can customising by adding in their own imagery, and content. The templates include all the files such as the \gls{HTML}, \gls{CSS}, and JavaScript files required for the templates to run smoothly. 

Template's simplify the web development process, by making it easy for yyythe end user who have little or no programming experience to build their own websites.

Taylor'd UI offers three templates for the end user to utilise in their learning of \gls{CSS}; a blog, portfolio, and product website. 

While each of the template utilise the framework as their foundations. Each template also have their own \gls{CSS} files with code that is unique to the template such as the blog template has a Read More button that expands the accompanying text section. The portfolio template has text that appears as an overlay when the user hovers over the image.

The blog template follows a simple theme, a jumbotron that has the title, and subtitle of the blog, followed by a blog post. The blog post has a title, and the date published at the top followed by an image. All the placeholder imagery is hosted by placehold.it \citep{PHI17}. 

By using an online image hosting server, the template files are smaller in size, links to the images are also less likely to break. Underneath the image, a paragraph of text is displayed, the user can click on the read more / read less button for more or less text to appear. The blog publisher and number of comments is the last section of the post. 