\newpage
\part{Development}
\chapter*{Framework}
\addcontentsline{toc}{chapter}{Framework}
The following section will detail the components that make up the Taylor'd UI framework. Each of the following components are what makes up the framework, and aid in the development of a website. All the components of the framework have been built in separate partial files, denoted by an underscore. The underscore also tells codekit, the compiler been used to not compile the files into their own css file. A separate file called taylord.scss was created, in this file the @include statement was used to pull all the separate partial files into one file and that file is then compiled to CSS. 

\subsection{Variables}
At the start of framework development, a file called \_foundation.scss was created, in this file all the variables for the framework are defined. In this file, the foundation colour variable is set and using the HSL function called lighten, a range of colours are created to be used in the framework. Other colour variables were added and then used to set the background colour, alert colours, and link colours.

\begin{lstlisting}[language=CSS3]
	$foundation-color: #000;
	$foundation-color-2121: lighten($foundation-color, 
	13%);

	h1, h2, h3, h4, h5, h6 {
  	color: $foundation-color-2121;
\end{lstlisting}

\subsection{Font and Typography}
In keeping with the framework being non opinionated, it was crucial to pick a font, and to have the typography nondescript so that the end user can change it to suit their needs. A range of fonts were looked at, the readability of the font in regards to large sections to single lines of text was looked at. Next, viewing the font on a range of devices ranging from mobile to desktop was looked at ensuing it was legible across these devices. The last aspect that was looked at was the range of the font, did it allow for non latin characters, accents and symbols. 

The font that passed these tests, and was a joy to look at was Open Sans. The next step was to determine the font sizing and how to calculate it. When decided on the best way to execute the font sizing, two methods were looked at, em and rem. Pixels was not looked at as early versions of Internet Explorer are not able to change the font size using browser functionality, a major usability issue. The em technique alters the base font size on the body element by using a percentage. 

This adapts the font so that 1em is equal to 10 pixels, instead of the default 16 pixels. To change the font size to the equivalent of 14 pixels, the em needs to be 1.4em. The downside of using em to calculate the font sizing is that the font size compounds. This means that a list within a list isn't 14 pixels but rather 20 pixels. There is a work around where any child elements are declared to use 1em, but an entry level user would not know this. 

With the advent of CSS3, rem which means root element was added, as previously mentioned, em sizing was relative to the font size of the parent whereas rem is relative to the root or html element. This means that a single font size can be defined for the html element, and all rem units will be a percentage of that base unit. Safari 5, Chrome, Firefox 3.6+, and even Internet Explorer 9 have support for rem units. Opera up to version 11.10, and early versions of Internet Explorer have yet to implement rem units. In order to display font on these browsers, a fallback pixel size is calculated using a mixin. 


http://www.gunlaug.no/contents/wd_additions_13.html

https://en.blog.wordpress.com/2012/10/09/open-sans-how-do-we-love-thee-let-us-count-the-ways/


\subsection{Colour Scheme}
When creating the colour scheme, it was important to keep design opinions to a minimum, to use colours that would get the frameworks point across. It was also key to use colours that the end user would recognise that they were for instructional purposes, and for them to change in their web development projects. The colour needed to look good when modified as well, this means if the hue, tone and vibrancy of the colour is modified, the resulting colour needed to look well. 

Instead of researching colour theory and choosing the best colours, it was decided to use Google's Material Design colour palette, and choose from their wide range of colours. The palette  gave a variety of colours along with modifications of the colour such as different hues and saturations. The colours selected to use in the framework were kept to primary colours to keep it as non opinionated as possible.


https://material.io/guidelines/style/color.html#color-color-tool


\subsection{Tables}

Tables in HTML should only be used for rendering data that naturally belongs in a grid based system. This is data where the data characterised is similar across a number of objects. Tables should not be used for the layout of content in a website, divs should be used for this. The key to designing the tables was to demonstrate to the end user that the tables were for data, and not for layout.

Five table variations were created for the framework, ranging from default table to striped tables, and included is table modifiers. Table modifiers take the colours that are used to dictate success or warning, and add them to a row in the table. The tables were also designed to be responsive, to achieve this the padding of table is calculated by dividing the \$baseline height by a set value. 

\begin{lstlisting}[language=CSS3]
    .table.table-condensed > thead > tr > th,
    .table.table-condensed > tbody > tr > td {
      padding-top: $baseline-size / 2.4; // 5px
      padding-bottom: $baseline-size / 2.4;  // 5px
      padding-left: $baseline-size / 1.5;  // 8px
      padding-right: $baseline-size / 1.5; // 8px
    }
\end{lstlisting}


\subsection{Buttons and Button Groups}
Buttons are an integral part of a framework. The styling, and the functionality of the buttons are key in the end users goal of using a website. If a button does not look like a button or if the styling of a button is overdone, the user can get confused, and not know how to proceed. With this in mind, the development of the buttons continued throughout the development of the framework. Originally, the buttons had round corners but this was removed in trying to keep the design non opinionated. 

Four button types were designed at the start of the project, these buttons are the basic buttons a developer would expect, and have minimal styling. From there 

\subsection{Grid}
\subsection{Alerts}
\subsection{Mixins}
\subsection{Labels}
\subsection{States}
\subsection{Panels}
\subsection{Tabs}

\newpage
\chapter*{Project Plan}
\addcontentsline{toc}{section}{Project Plan}
add stuff about what was built each week - similar to the engineering thing 